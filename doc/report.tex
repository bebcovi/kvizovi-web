\documentclass{scrreprt}

\usepackage[croatian]{babel}
\usepackage[utf8]{inputenc}
\usepackage[T1]{fontenc}
\usepackage{hyperref}
\usepackage{graphicx}
\usepackage{tikz}
\usepackage{pgfplots}
\usepackage{float}
\usepackage[nottoc,numbib]{tocbibind}
\usepackage{color}

\setlength{\parskip}{\bigskipamount}
\setlength{\parindent}{0pt}

\graphicspath{ {./images/} }

\begin{document}

\titlehead{Sveučilište u Zagrebu\\Filozofski fakultet\\Odsijek za informacijske
i komunikacijske znanosti\\Akademska godina 2013/14.}
\title{Upotreba web aplikacije za kvizove u nastavi osnovnih i srednjih škola}
\author{Studenti: Janko i Matija Marohnić\\Mentor: prof. dr. sc. Kristina
Kocijan}
\date{Zagreb, 2014.}

\maketitle

\pagebreak

Ovaj rad izrađen je na Filozofskom fakultetu u Zagrebu pod vodstvom prof. dr.
sc. Kristine Kocijan i predan je na natječaj za dodjelu Rektorove nagrade u
akademskoj godini 2013/2014.

\pagebreak

\tableofcontents

\chapter{Uvod}

Kvizovi su aplikacija za riješavanje kvizova koja je prilagođena osnovnim i
srednjim školama tako da se u korijenu dijeli na dvije uloge -- učenike i
profesore.

Postoje razni kreativni načini da se upotrijebi informacijska tehnologija u
nastavi, ali to je obično prilično jednostrano. Smatramo da se trebaju razvijati
puno interaktivnije i zabavnije metode podučavanja. U ovom smo istraživanju
testirali aplikaciju za kvizove koju razvijamo na nekoliko osnovnih i srednjih
škola. Cilj ovog istraživanja bio je saznati je li takav način učenja pomaže
učenicima da bolje savladaju gradivo, ali i čini li aplikacija profesorima
podučavanje zanimljivijim.

Hipoteza našega istraživanja je da učenici koji su učili riješavajući kvizove
pomoću aplikacije bolje usvajaju gradivo od onih koji nisu.

\chapter{Sredstva i metode}

Istraživanje smo proveli tako što smo izradili aplikaciju za kvizove i dali
određenom uzorku osnovnih i srednjih škola na testiranje. Korisnik započinje
korištenje aplikacije tako da odabere svoju ulogu u školi, stoga ovisno o ulozi
aplikacija pruža drugačiju funkcionalnost.

Podatke o kvizovima skupljali smo 2 godine. U svakom razredu su učenici bili
podijeljeni na skupinu koja ne riješava kvizove i na onu koja riješava, na taj
način smo mogli provjeravati učinkovitost aplikacije kada bi učenici imali test
znanja. Imamo puno podataka na raspolaganju, ali smo za svaki slučaj proveli još
i ankete kako bismo dobili podatke koje pasivno ne bismo mogli dobiti. Anketa za
učenike izgledala je ovako:

\textcolor{red}{<anketa za učenike>}

a anketa za škole je izgledala ovako:

\textcolor{red}{<anketa za škole>}

Imamo blog na kojemu obaviještavamo korisnike o ažuriranju aplikacije. S obzirom
da je dio ciljane publike starija populacija koja nije nužno iskusna s
aplikacijama, dodali smo i vodič kroz aplikaciju koji je dostupan u bilo kojem
trenutku, kako bi se korisnici lakše snašli. Također, u slučaju da ima nekih
problema, korisnici nas uvijek mogu kontaktirati putem e-mail adresa koje se
nalaze u kontakt sekciji koju smo stavili u glavnu navigaciju. Korisnici također
mogu naknadno izmjenjivati svoje podatke u korisničkom računu.

\section{Škola}

\emph{Škola} je uloga koja predstavlja profesora i ona je administrator kvizova.
Nakon odabira te uloge korisnik se može prijaviti ili registrirati ako još nema
korisnički račun. Registracija se sastoji od ispunjavanja jednostavnog formulara
pomoću kojega skupljamo informacije o korisnicima koje možemo iskoristiti kako
bismo poboljšali njihovo iskustvo i kako bismo mogli raditi istraživanja.
Zanimljivo polje u formularu za registraciju je \emph{Tajni ključ}, koji je
potreban za registraciju učenicima te škole.

Nakon prijave profesore dočeka lista kvizova koje su napravili, gdje mogu
izmjenjivati postojeće kvizove i sastavljati nove. Nakon što je profesor
zadovoljan s kvizom, može ga učiniti aktivnim, odnosno vidljivim učenicima.
Izmjenjivanje kvizova podijeljeno je na izmjenu metapodataka kviza i na izmjenu
pitanja kviza. Postoji 4 vrsta pitanja:

\begin{itemize}
  \item Točno/netočno
  \item Ponuđeni odgovori
  \item Asocijacija
  \item Upiši točan odgovor
\end{itemize}

Uz svako se pitanje može pridružiti pomoć, koja će se prikazati učenicima dok
riješavaju pitanje.

\section{Učenik}

\emph{Učenik} je uloga koja riješava kvizove koje je napravila njihova škola.
Kao i kod škole, učenik se može prijaviti ili registrirati, ako već nema
korisnički račun. Pri registraciji učenik treba napisati tajni ključ koji mu je
njegova škola dala, u protivnom se ne može registrirati. Na taj način
spriječavamo da se bilo tko registrira kao učenik. Nakon prijave ili
registracije, korisnika dočeka lista kvizova koji su dostupni za riješavanje.
Kviz je moguće igrati sam ili u paru, u drugom slučaju se drugi igrač također
treba prijaviti. Nakon što učenik započne kviz, prikazuje mu se jedno po jedno
pitanje na koja treba odgovoriti. Pitanja su obično vremenski ograničena, tako
da se učenik ne stigne previše konzultirati s vanjskim izvorima, kako bi igra
bila što poštenija. Nakon što učenik odgovori na sva pitanja, ispisuju se
rezultati i učenik dobiva određenu ``titulu'' s obzirom na njegov rezultat.
Titule su osmišljene tako da se učenika uvijek pohvaljuje, čak i ako je imao loš
rezultat, tako da se učenik dobro osjeća i da ga se potiče da igra i dalje.

\chapter{Rezultati}

\chapter{Tablice i grafovi}

\chapter{Rasprava}

Tijekom izrade ove aplikacije uočili smo mnogo načina na koji bismo mogli
poboljšati aplikaciju, već imamo mnogo promijena na putu. Jedna od dramatičnijih
promijena koju smo odlučili napraviti je učiniti aplikaciju dostupnu svima, ne
samo učenicima i profesorima, gdje bi svi mogli izrađivati kvizove i riješavati
ih. Takvo bi otvorenje zahtijevalo oprez i moglo bi promijeniti karakter
aplikacije, ali mislim da bi s tom promijenom bilo puno zanimljivije koristiti
aplikaciju.

Kako bi kvizovi dobili malo više osobnosti, bit će moguće pridružiti im slike.

Odlučili smo ukloniti vodič kroz aplikaciju jer ju želimo napraviti dovoljno
jednostavnom da upute nisu potrebne, već da je jasno samo po sebi kako postići
određeni cilj.

\chapter{Zahvale}

Htjeli bismo zahvaliti našim mentorima, prof. dr. sc. Kristini Kocijan i
\textcolor{red}{<ime Jankovog mentora>}, za veliku podršku u ovom projektu i
prikupljanju uzorka škola koje su voljne koristiti našu aplikaciju i pomoći nam
kod istraživanja. Nadalje, htjeli bismo zahvaliti profesorima i učenicima za
sudjelovanje u ovom istraživanju i za slanje prijedloga i prijavu grešaka, što
nam je puno pomoglo kod unaprijeđivanja aplikacije.

Sortirane prema broju aktivnih kvizova, škole koje su sudjelovale u istraživanju
su: I. gimnazija, Osnovna škola Bogumila Tonija, Osnovna škola Stjepana Radića,
Srednja škola Metković, Osnovna škola Poreč, Osnovna škola Stubičke Toplice,
Gimnazija Sesvete, Poljoprivredna škola, Srednja škola ``Ivan Seljanec''
Križevci, Gimnazija fra Dominika Mandica, Medicinska škola Varaždin, Srednja
škola Čakovec, Tehnička škola Šibenik, Privatna srednja ekonomska škola INOVA,
OŠ KV, Škola za umjetnost, dizaj, grafiku i odjeću, Zabok, Gimnazija i strukovna
škola Jurja Dobrile, Nadbiskupska klasična gimnazija.

\begin{thebibliography}{9}
  \bibitem{lamport94} Leslie Lamport, \emph{\LaTeX: A Document Preparation
    System}. Addison Wesley, Massachusetts, 2nd Edition, 1994.
\end{thebibliography}

\end{document}
