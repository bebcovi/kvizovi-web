\documentclass{scrreprt}

\usepackage[croatian]{babel}

\usepackage[utf8]{inputenc}

\usepackage[T1]{fontenc}

\usepackage{hyperref}

\usepackage{graphicx}

\usepackage{tikz}

\usepackage{pgfplots}

\usepackage{float}

\setlength{\parskip}{\bigskipamount}

\setlength{\parindent}{0pt}

\graphicspath{ {./images/} }

\begin{document}

\titlehead{Sveučilište u Zagrebu\\Filozofski fakultet\\Odsijek za informacijske
i komunikacijske znanosti\\Akademska godina 2013/14.}

\title{Upotreba web aplikacije za kvizove u nastavi osnovnih i srednjih škola}

\author{Studenti: Janko i Matija Marohnić\\Mentor: prof. dr. sc. Kristina
Kocijan}

\date{1. srpnja 2014.}

\maketitle

\tableofcontents

\chapter{Sažetak}

\chapter{Uvod}

Kvizovi su aplikacija za riješavanje kvizova koja je prilagođena osnovnim i
srednjim školama tako da se u korijenu dijeli na dvije uloge -- učenike i
profesore.

Postoje razni kreativni načini da se upotrijebi informacijska tehnologija u
nastavi, ali to je obično prilično jednostrano. Smatramo da se trebaju razvijati
puno interaktivnije i zabavnije metode podučavanja. U ovom smo istraživanju
testirali aplikaciju za kvizove koju razvijamo na nekoliko osnovnih i srednjih
škola. Cilj ovog istraživanja bio je saznati je li takav način učenja pomaže
učenicima da bolje savladaju gradivo, ali i čini li aplikacija profesorima
podučavanje zanimljivijim.

\chapter{Sredstva i metode}

Istraživanje smo proveli tako što smo izradili aplikaciju za kvizove i dali
određenom uzorku osnovnih i srednjih škola na testiranje. Aplikacija počinje
direktnim odabirom uloga \cite{lamport94}

\chapter{Rezultati}

\chapter{Tablice i grafovi}

\chapter{Rasprava}

\chapter{Zahvala}

\begin{thebibliography}{9}

  \bibitem{lamport94} Leslie Lamport, \emph{\LaTeX: A Document Preparation
    System}. Addison Wesley, Massachusetts, 2nd Edition, 1994.

\end{thebibliography}

\end{document}
