\documentclass{scrreprt}

\usepackage[croatian]{babel}
\usepackage[utf8]{inputenc}
\usepackage[T1]{fontenc}
\usepackage{hyperref}
\usepackage{graphicx}
\usepackage{tikz}
\usepackage{pgfplots}
\usepackage{float}
\usepackage[nottoc,numbib]{tocbibind}

\setlength{\parskip}{\bigskipamount}
\setlength{\parindent}{0pt}

\graphicspath{ {./images/} }

\begin{document}

\titlehead{Sveučilište u Zagrebu\\Filozofski fakultet\\Odsijek za informacijske
i komunikacijske znanosti\\Akademska godina 2013/14.}
\title{Upotreba web aplikacije za kvizove u nastavi osnovnih i srednjih škola}
\author{Studenti: Janko i Matija Marohnić\\Mentor: prof. dr. sc. Kristina
Kocijan}
\date{1. srpnja 2014.}

\maketitle
\tableofcontents

\chapter{Sažetak}

\chapter{Uvod}

Kvizovi su aplikacija za riješavanje kvizova koja je prilagođena osnovnim i
srednjim školama tako da se u korijenu dijeli na dvije uloge -- učenike i
profesore.

Postoje razni kreativni načini da se upotrijebi informacijska tehnologija u
nastavi, ali to je obično prilično jednostrano. Smatramo da se trebaju razvijati
puno interaktivnije i zabavnije metode podučavanja. U ovom smo istraživanju
testirali aplikaciju za kvizove koju razvijamo na nekoliko osnovnih i srednjih
škola. Cilj ovog istraživanja bio je saznati je li takav način učenja pomaže
učenicima da bolje savladaju gradivo, ali i čini li aplikacija profesorima
podučavanje zanimljivijim.

Hipoteza našega istraživanja je da učenici koji su učili riješavajući kvizove
pomoću aplikacije bolje usvajaju gradivo od onih koji nisu.

\chapter{Sredstva i metode}

Istraživanje smo proveli tako što smo izradili aplikaciju za kvizove i dali
određenom uzorku osnovnih i srednjih škola na testiranje. Aplikacija počinje
direktnim odabirom uloga \cite{lamport94}

\chapter{Rezultati}

\chapter{Tablice i grafovi}

\chapter{Rasprava}

Tijekom izrade ove aplikacije uočili smo mnogo načina na koji bismo mogli
poboljšati aplikaciju, već imamo mnogo promijena na putu. Jedna od dramatičnijih
promijena koju smo odlučili napraviti je učiniti aplikaciju dostupnu svima, ne
samo učenicima i profesorima, gdje bi svi mogli izrađivati kvizove i riješavati
ih. Takvo bi otvorenje zahtijevalo oprez i moglo bi promijeniti karakter
aplikacije, ali mislim da bi s tom promijenom bilo puno zanimljivije koristiti
aplikaciju.

Kako bi kvizovi dobili malo više osobnosti, bit će moguće pridružiti im slike.

Odlučili smo ukloniti vodič kroz aplikaciju jer ju želimo napraviti dovoljno
jednostavnom da upute nisu potrebne, već da je jasno samo po sebi kako postići
određeni cilj.

\chapter{Zahvale}

Htjeli bismo zahvaliti našim mentorima, prof. dr. sc. Kristini Kocijan i
\textcolor{red}{<ime Jankovog mentora>}, za veliku podršku u ovom projektu i
prikupljanju uzorka škola koje su voljne koristiti našu aplikaciju i pomoći nam
kod istraživanja. Nadalje, htjeli bismo zahvaliti profesorima i učenicima za
sudjelovanje u ovom istraživanju i za slanje prijedloga i prijavu grešaka, što
nam je puno pomoglo kod unaprijeđivanja aplikacije.

Sortirane prema broju aktivnih kvizova, škole koje su sudjelovale u istraživanju
su: I. gimnazija, Osnovna škola Bogumila Tonija, Osnovna škola Stjepana Radića,
Srednja škola Metković, Osnovna škola Poreč, Osnovna škola Stubičke Toplice,
Gimnazija Sesvete, Poljoprivredna škola, Srednja škola ``Ivan Seljanec''
Križevci, Gimnazija fra Dominika Mandica, Medicinska škola Varaždin, Srednja
škola Čakovec, Tehnička škola Šibenik, Privatna srednja ekonomska škola INOVA,
OŠ KV, Škola za umjetnost, dizaj, grafiku i odjeću, Zabok, Gimnazija i strukovna
škola Jurja Dobrile, Nadbiskupska klasična gimnazija.

\begin{thebibliography}{9}
  \bibitem{lamport94} Leslie Lamport, \emph{\LaTeX: A Document Preparation
    System}. Addison Wesley, Massachusetts, 2nd Edition, 1994.
\end{thebibliography}

\end{document}
